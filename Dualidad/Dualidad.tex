% Created 2019-05-17 vie 12:17
% Intended LaTeX compiler: pdflatex
\documentclass[11pt]{article}
\usepackage[utf8]{inputenc}
\usepackage[T1]{fontenc}
\usepackage{graphicx}
\usepackage{grffile}
\usepackage{longtable}
\usepackage{wrapfig}
\usepackage{rotating}
\usepackage[normalem]{ulem}
\usepackage{amsmath}
\usepackage{textcomp}
\usepackage{amssymb}
\usepackage{capt-of}
\usepackage{hyperref}
\author{Rubén Zavala Díaz}
\date{17 de mayo del 2019}
\title{Dualidad en programación Lineal}
\hypersetup{
 pdfauthor={Rubén Zavala Díaz},
 pdftitle={Dualidad en programación Lineal},
 pdfkeywords={},
 pdfsubject={},
 pdfcreator={Emacs 25.2.2 (Org mode 9.2.3)}, 
 pdflang={English}}
\begin{document}

\maketitle
\tableofcontents


\section{Introducción}
\label{sec:org389be92}
La Dualidad es una manera de asociar un cierto problema de
programación lineal a cada problema de programacón lineal
\section{Ejemplo}
\label{sec:org7bca6a7}

Consideremos el siguiente problema de Programaón Lineal:
\begin{equation*}
 \begin{aligned}
 \text{Maximizar} \quad & 2x_{1}+3x_{2}\\
 \text{sujeto a} \quad &
   \begin{aligned}
    4x_{1}+8x_{2} &\leq 12\\
    2x_{1}+x_{2} &\leq 3\\
    3x_{1}+2x_{2} &\leq 4\\
    x_{1} &\geq  0\\
    x_{2} &\geq 0
   \end{aligned}
 \end{aligned}
 \end{equation*}

\begin{itemize}
\item Podemos observar que bajo las restricciones la función objetivo es menor que 12 ya que:
\end{itemize}
\begin{equation*}
     zx_{1}+3x_{2}\leq 4x_{1}+8x_{2}\leq 12.
\end{equation*}
\begin{itemize}
\item Siguiente paso\ldots{}
\end{itemize}
\section{Teoremas}
\label{sec:org010a598}
\end{document}